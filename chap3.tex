 
 \chapter{LITERATURE SURVEY}
 
 \section{Refered Paper 1}
 \begin{itemize}
 	\item  \Large\textbf {Design of Automatic Hand Sanitizer System Compatible with Various Containers.}
 \end{itemize}
 { \Large \textbf{Abstract:}}
 Demand for hand sanitizers has surged since the coronavirus broke out and spread around the world. Hand sanitizers are usually applied by squirting the sanitizer liquid when one presses a pump with one’s hand. This causes many people to come into contact with the pump handle, which increases the risk of viral transmission. Some hand sanitizers on the market are automatically pumped. However, because sanitizer containers and pump devices are designed to be compatible only between products produced by the same manufacturer, consumers must also repurchase the container for the liquid if they replace the hand sanitizer. Therefore, this paper suggests the design of an automatic hand sanitizer system compatible with various sanitizer containers.

 
 
  \newpage
 \section{Refered Paper 2}
 \begin{itemize}
 	\item  \Large\textbf{Bidirectional Visitor Counter using Arduino.}
 	
 \end{itemize}
 
 { \Large \textbf{Abstract:}}
 It is a circuit used for accurately counting the number of persons/visitors entering or leaving the premises. if somebody enters the premises then the Counter is incremented by one, or decremented by one if someone leaves the premises. This count will be very accurate. The aggregate number of people will appear on the 16X2 LCD module. In the circuit an Arduino UNO Board is utilized. This will help in the accurate measurement of the visitors and is less complex compared to a microcontroller. The Arduino receives signals from the sensors and operate under the control of program stored in Arduino rom. there are two IR modules one at the entrance and other at the exit gate to count the number of persons entering and leaving the premises respectively. The main concept of this system is to keep track of people present inside the premises which is very useful in current situation.
 

\newpage
\section{Refered Paper 3}
 \begin{itemize}
 	\item  \Large\textbf {	Automatic Water Tank Filling System Controlled Using ArduinoTM Based Sensor for Home Application.}
 \end{itemize}
 { \Large \textbf{Abstract:}}
 Water supply is the most important thing in daily home activity especially for washing, cleaning, and taking a bath. The Indonesian villagers commonly supply the water by pumping the groundwater to fill a water tank. However, the utilization of non-automated switch used to turn on and turn off a pumping machine sometimes causes either the water spills or a wasteful electrical consumption. The previous works reported the utilizations of ArduinoTM based sensors for plant watering system, water tank overflow control, and automated irrigation system. In this work, an automated water tank filling system will be proposed. The system is designed by applying an ultrasonic sensor, an automatic switch module, a water-flow sensor, an ArduinoTM microcontroller, and a pumping machine in order to automatically switch the water filling. By applying an ultrasonic sensor, an ultrasonic transmitter is mounted on the top of the tank and transmits an ultrasonic pulse down into the tank. This pulse which travels at the speed of sound will be reflected back to the transmitter from the liquid surface. The time delay measurement between transmitted and received signals enables the device to calculate the distance to the surface. The transmitter is programmed to automatically determine the liquid level and switch the pumping machine. The dynamics of water flow and liquid level during filling and draining the water tank will be reported. We hope to this system, people will enjoy supplying water without their worries related to water spills and a wasteful electrical consumption.

%Note9 With a little pactice of this code you will get the idea about how to 
% use $ $, \[  \],  \\ ,
% how to insert graphics and  create Tables.
%Note10: If u are creating table of contents, List of Figures,cross reference, Citation, ...,
%then run the same code for two times.